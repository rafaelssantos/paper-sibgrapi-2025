\section{Method}
\label{sect:method}

This section presents our unsupervised method for transfer function (TF) design, enabling semi-automated material classification and initial TF specification to support intuitive volume exploration.

An overview is shown in Fig.~\ref{fig:volume-exploration-pipeline}. After organizing the multidimensional data into a volume grid, our method comprises three main steps: dimensionality reduction, clustering and representative selection. 

The techniques used at each step were carefully chosen based on their time complexity, ensuring a balance between computational efficiency and effectiveness for handling large volume datasets. This design choice supports practical scalability and responsiveness, which are critical for interactive volume exploration.


\begin{figure*}[htb!]
    \centering
    \caption{Overview of the proposed unsupervised method for transfer function definition and design.}
    \label{fig:volume-exploration-pipeline}
    \includegraphics[width=\textwidth]{figs/method-overview.jpg}
\end{figure*}

\subsection{Dimensionality reduction}
\label{subsect:feature-extraction}

Dimensionality reduction is a critical step for two reasons. First, the clustering technique employed requires a two-dimensional input space for proper functioning (see Section~\ref{subsect:clustering}). Second, our TF design interface is two-dimensional (see Section~\ref{sect:volume-exploration}).

We adopt FastMap~\cite{faloutsos1995} to project high-dimensional voxel data into 2D while preserving data similarity. FastMap operates by selecting two distant pivots and projecting all points onto the line defined by these pivots, recursively reducing the dimensionality.

Let $d$ be the number of attributes and $n$ the number of voxels. The algorithm proceeds as follows:

\begin{enumerate}
    \item Select two points with maximal pairwise distance as the pivots.
    \item Project all points onto a hyperplane orthogonal to the line defined by the pivots.
\end{enumerate}

To mitigate the computational cost of pivot selection, \cite{faloutsos1995} proposed the approach summarized in Algorithm~\ref{alg:pivot-searching-of-fastmap}.

\begin{algorithm}
    \caption{Pivot searching in FastMap.}
    \label{alg:pivot-searching-of-fastmap}
    \KwIn{$\mathbb{O}$}
    \KwOut{Pivots $O_a$, $O_b$}
    $O_a \gets$ random point $o \in \mathbb{O}$\\
    $O_b \gets$ point $o \in \mathbb{O}$ farthest from $O_a$\\
    $O_a \gets$ point $o \in \mathbb{O}$ farthest from $O_b$
\end{algorithm}

\subsection{Clustering}
\label{subsect:clustering}

To simplify material classification and enhance the detection of relevant volume structures, we employ the classical density-based clustering technique DBSCAN~\cite{ester1996}.

DBSCAN is widely recognized for its effectiveness in identifying clusters of arbitrary shapes without requiring prior knowledge of the number of clusters~\cite{schubert2017}. However, its standard implementation has a worst-case time complexity of $\mathcal{O}(n^2)$~\cite{schubert2017}. To ensure practical scalability, we adopt an optimized grid-based variant~\cite{gunawan2013}, which reduces the time complexity to $\mathcal{O}(n \log n)$.

As in the original algorithm, two parameters must be tuned by the user: $minPts$, the minimum number of points to form a dense region, and $\varepsilon$, the neighborhood radius.

At the end of this step, each cluster contains a subset of voxels potentially representing distinct features of interest (FOI) within the volume.

\subsection{Representative selection}
\label{subsect:representative-selection}

The voxels classified by DBSCAN could be directly projected onto the TF design interface. However, to reduce clutter in the scatter plot and improve interpretability, we apply a representative selection technique within each cluster.

First, representative pivots are selected using Sparse Spatial Selection (SSS)~\cite{pedreira2007}. This technique iteratively adds points as pivots if they are sufficiently distant from all previously selected pivots, controlled by a distance factor $\alpha$ (Algorithm~\ref{alg:sss}).

\begin{algorithm}
    \caption{Sparse Spatial Selection (SSS).}
    \label{alg:sss}
    \KwIn{Points $\mathbb{P}$}
    \KwOut{Selected pivots $\mathbb{P}_s$}
    $\mathbb{P}_s \gets \{p_1\}$ \\
    \ForEach{$p \in \mathbb{P}$}{
        \If{$\forall p_s \in \mathbb{P}_s, \text{dist}(p, p_s) \geq M\alpha$}{
            $\mathbb{P}_s \gets \mathbb{P}_s \cup \{p\}$
        }
    }
\end{algorithm}

Next, each cluster is subdivided into sub-clusters by assigning every point to its nearest pivot (Algorithm~\ref{alg:subclustering-finding}). This step refines the initial data classification and acts as a second-level clustering, improving the granularity of the FOI representation.

\begin{algorithm}
    \caption{Sub-cluster assignment within a cluster.}
    \label{alg:subclustering-finding}
    \KwIn{Points $\mathbb{P}$ of cluster $c$}
    \KwIn{Pivots $\mathbb{P}_s$ of cluster $c$}
    \KwOut{Points with sub-cluster assignments}
    \ForEach{$p \in \mathbb{P}$}{
        $p_s \gets$ nearest pivot in $\mathbb{P}_s$ \\
        Assign $p$ to $p_s$'s sub-cluster
    }
\end{algorithm}

The parameter $\alpha$ controls the number of selected pivots: smaller values lead to more pivots and finer sub-clustering, while values closer to $1$ yield fewer, broader sub-clusters.
