\section{Introduction}
\label{sect:introduction}

Direct Volume Rendering (DVR) is a powerful technique used in computer science to visualize three-dimensional scalar data grids, particularly in scientific and medical applications~\cite{elvins1992, xu2021}. A central component of the DVR pipeline—and the primary focus of this work—is the transfer function (TF), which maps volume data (e.g., density) to visual properties such as color and opacity~\cite{ljung2016}.

When interacting with a DVR system, users often adjust TF parameters to reveal specific regions of interest within the dataset. A TF is a mathematical function whose input consists of volume data attributes. The accuracy of the resulting classification is directly influenced by the chosen data domain. Several studies~\cite{ljung2016, cai2017, pfister2001, roettger2005} have demonstrated that multidimensional TFs can significantly enhance discriminative power. However, simply increasing the number of input attributes does not necessarily lead to better classification. Since no universal TF suits all datasets, TF design is typically left to the user's expertise and understanding of the data domain~\cite{arens2010}.

Even after defining the data domain, adjusting TF parameters remains essential to highlight desired volume features. TF design is inherently non-intuitive in one-dimensional spaces~\cite{pfister2001, wang2011}, and this complexity increases in higher-dimensional settings. While material classification benefits from additional dimensions, specifying TFs in such spaces is notoriously difficult~\cite{ljung2016, pfister2001, kniss2002, pan2024}.

We propose a low-computational-cost method that simplifies TF design, regardless of the dimensionality of the data domain. Our approach follows an unsupervised learning perspective, integrating clustering, dimensionality reduction, and pivot-based indexing. We also introduce an exploration scheme that enables users to navigate a set of volume features that are semi-automatically classified and mapped onto a modified 2D scatter plot view.

Our main contributions are as follows:

\begin{itemize}
    \item A low-cost and effective TF design approach. We leverage semi-automated material classification to generate TFs that require minimal parameter adjustment.
    \item An intuitive volume exploration scheme. We provide a user-friendly scatter plot interface for navigating the classified data.
\end{itemize}

The remainder of this paper is organized as follows. Section~\ref{sect:related-works} reviews related work. Section~\ref{sect:method} describes the proposed method. Section~\ref{sect:volume-exploration-space} details the volume exploration scheme and TF design interface. Section~\ref{sect:results} presents the results, followed by a discussion in Section~\ref{sect:discussion}. Finally, Section~\ref{sect:conclusions} concludes the paper.
