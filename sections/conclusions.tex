\section{Conclusions}
\label{sect:conclusions}
In this work, we presented a robust method for TF definition and design from an unsupervised data perspective. Our comprehensive approach covers the entire classification process, from feature selection to establishing a data domain, developing a TF design interface, and creating a simplified volume exploration space. We proposed a heuristic for feature selection based on similarity rankings of attributes,  employed FastMap for efficient feature extraction, utilized DBSCAN for effective clustering, and leveraged SSS for pivot-based indexing. These techniques collectively facilitate semi-automatic classification and initial TF specification, forming the basis of our TF design interface and exploration system, as demonstrated through a scatter plot view.

The method exhibits minimal computational overhead, a brief runtime, and low storage requirements, highlighting its practicality and scalability for real-world applications. However, the current feature selection process, heavily reliant on user input, presents a limitation due to potential variability introduced by differing levels of user expertise. Nevertheless, the results are satisfactory in all experiments, given the unlabeled nature of the data. To address this, future work will focus on investigating advanced feature selection techniques and other stop criteria.

Another point of investigation is subjecting the method to handling large and high-dimensional datasets. The techniques included in the method are capable of operating well on datasets with these characteristics, and further practical investigation is needed in future studies to confirm its effectiveness.

Furthermore, we recognize the necessity to evaluate our proposed method using multivariate data, aiming to expand its applicability and robustness across diverse datasets.