\section{Related Works}
\label{sect:related-works}

Various aspects of TFs have been extensively discussed in the literature~\cite{ljung2016}. Our review focuses on strategies for managing the complexity of defining multidimensional TFs, particularly those involving machine learning, dimensionality reduction, and information visualization.

Histograms are frequently used in 2D TF design, typically representing intensity--gradient magnitude. Automated techniques often combine histograms with clustering algorithms such as affinity propagation~\cite{zhang2016}, hierarchical clustering~\cite{sereda2006}, and iterative self-organizing methods~\cite{tzeng2004}. \citet{roettger2005} proposed classifying spatially connected regions using gradient and coordinate information.

Multidimensional TF design commonly follows two strategies: (i) interactive interfaces enabling direct manipulation of attributes, such as parallel coordinate plots (PCP); and (ii) dimensionality reduction techniques like MDS or PCA, producing simplified visual representations~\cite{tory2005, zhao2010, guo2011}.

Self-Organizing Maps (SOM) have been used to reduce dimensionality and create interactive maps~\cite{moura2007}. Extensions include spherical SOMs~\cite{khan2015}, hierarchical clustering with modified dendrograms~\cite{wang2011}, and normalized cuts forming cell maps~\cite{cai2017}. Our method builds on~\cite{cai2017} by employing MDS and density-based clustering to automate material classification.

Supervised learning has also been explored: neural networks and SVMs~\cite{tzeng2005}, SOM with backpropagation~\cite{wang2006}, and deep learning approaches using GANs and CNNs for TF generation and visualization~\cite{berger2018, hong2019, kim2021, pan2024}. \citet{sharma2020} proposed a graph-based method to identify and highlight significant volume structures through feature clustering and topology analysis.
