\section{Related Works}
\label{sect:related-works}

Various aspects of transfer functions (TFs) have been extensively discussed~\cite{ljung2016}. We focus on methods that support user interaction in multidimensional TF design, especially those using machine learning, dimensionality reduction, and information visualization.

Multidimensional TFs incorporate multivariate or derived attributes. While multivariate data come directly from acquisition, derived features—such as gradient magnitude, curvature, or texture—are computed from primary data. Selecting relevant attributes is challenging due to the “curse of dimensionality.” Dimensionality reduction is a common solution~\cite{cai2017, abbasloo2016, gao2022, moura2007, zhao2010}.

Histograms are widely used in 2D TF design~\cite{kniss2002}, often representing intensity–gradient magnitude. Several methods automate histogram-based TFs by grouping spatial regions or combining with clustering algorithms such as affinity propagation~\cite{zhang2016}, hierarchical clustering~\cite{sereda2006}, and self-organizing maps~\cite{tzeng2004}.

TF design strategies typically follow two directions: user interfaces for direct manipulation of attributes (e.g., parallel coordinates)~\cite{tory2005, zhao2010}, or projection-based approaches using MDS or PCA~\cite{guo2011}. SOM-based methods~\cite{moura2007, khan2015, cai2017} perform dimensionality reduction and offer interactive maps of similar voxel regions. Our method shares this philosophy but uses an MDS-based projection and density-based clustering to produce a scatter plot interface.

Supervised learning has also been used in TF design, with approaches based on neural networks~\cite{tzeng2005, wang2006}, GANs~\cite{berger2018, hong2019}, and CNNs~\cite{kim2021, pan2024}. Graph-based strategies~\cite{sharma2020} enhance volume rendering by identifying and emphasizing key structures.
