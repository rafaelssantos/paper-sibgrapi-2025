\section{Related Works}
\label{sect:related-works}

Various aspects of transfer functions (TFs) have been extensively discussed in the literature~\cite{ljung2016}. Our review focuses on methods that support user interaction in multidimensional TF design, particularly those employing machine learning, dimensionality reduction, and information visualization techniques.

A typical multidimensional TF incorporates either multivariate or derived input data. Multivariate attributes originate from the volume acquisition process, while derived attributes are computed from primary data, such as density. The gradient is the most commonly used derived attribute, but others include curvature, size, distance, texture and statistical measures.

Selecting an optimal subset of attributes to maximize material classification accuracy is a complex task. \cite{arens2010} argue that no single TF design is universally effective. Considering all available attributes is impractical, as it may increase computational costs and introduce noise, degrading classification performance. This classical challenge is known as the ``curse of dimensionality.'' Dimensionality reduction is the most common approach to address this issue and several studies have applied such techniques to multidimensional TF design~\cite{cai2017, abbasloo2016, gao2022, moura2007, zhao2010}.

Histograms are commonly used as  componente for 2D TFs design~\cite{kniss2002}, typically representing intensity–gradient magnitude or low–high histograms. Several approaches have been proposed to automate histogram-based TF design. Röttger et al.~\cite{roettger2005} group spatially connected regions and associate gradient values with spatial coordinates to classify datasets. Many methods combine histograms with clustering algorithms, including affinity propagation~\cite{zhang2016}, hierarchical clustering~\cite{sereda2006}, and iterative self-organizing data analysis~\cite{tzeng2004}.

Approaches to multidimensional TF design generally follow two main strategies. The first provides an interface that allows users to manipulate all data attributes—such as the parallel coordinate plot (PCP). The second applies dimensionality reduction techniques, such as Multidimensional Scaling (MDS) or Principal Component Analysis (PCA), to create simplified visual representations.

\cite{tory2005} employed PCP in their exploration scheme, integrating viewer parameters and TF specification into the interface. \cite{zhao2010} followed a similar approach, applying a local linear embedding technique for dimensionality reduction. Likewise, \cite{guo2011} proposed a hybrid interface that combines PCP with a scatter plot generated using MDS.

\cite{moura2007} applied a Self-Organizing Map (SOM) and a radial basis function for TF design. SOM performs unsupervised learning to reduce dimensionality, producing a map in which neighboring regions represent similar voxels. Users interact with the map by drawing widgets in specific regions. \cite{khan2015} extended this idea using a spherical SOM, enabling interaction on a spherical lattice. \cite{wang2011} proposed a volume exploration space based on subtree structures derived from hierarchical clustering and modified dendrograms. Later, \cite{cai2017} revisited these ideas, augmenting SOM with a normalized cut step to create a “cell map,” in which each region encodes volume information tied to meaningful structures. Our method shares similarities with the work of \cite{cai2017}, but employs an MDS-based technique and a density-based clustering algorithm to automate material classification, resulting in a modified scatter plot view.

\cite{tzeng2005} proposed one of the earliest TF design strategies using supervised learning, implementing neural networks and support vector machines.\cite{wang2006} combined SOM with backpropagation neural networks for material classification. \cite{berger2018} used a Generative Adversarial Network (GAN) framework to compute models for TF specification and view selection. More recently, \cite{hong2019} combined GANs with Convolutional Neural Networks (CNNs) to synthesize the exploration process. \cite{kim2021} developed a method using CNNs to generate visualizations from TF colorizations. \cite{pan2024} introduced a deep learning-based gallery approach with differentiable rendering to support user exploration of the design space.

\cite{sharma2020} proposed a graph-based method for identifying significant volume structures, which involves clustering features, building a material graph topology, and enhancing the rendering of important structures.
